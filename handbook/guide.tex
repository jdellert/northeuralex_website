\documentclass[11pt, a4paper]{article}
\usepackage[margin=2.5cm]{geometry}

\title{\textbf{System Administration Handbook for NorthEuraLex}}
\author{
\date{March 2019}
}
\begin{document}

\maketitle

\section{Accessing the servers}

There are a few different servers for which the system administrator is responsible:

\begin{enumerate}
\item Satie
\begin{itemize}
\item Main entry point for the Xen bridge
\end{itemize}
\item Wagner
\item WWW2 (Vivaldi)
\item WWW2 (Vivaldi) Database
\item Beethoven
\item Strauss
\item Haendel
\item Haendel Fileserver
\item Haendel Mysql
\item Scarlatti
\end{enumerate}

\noindent
In order to access all servers except Satie, you need to specify the port of entry:

\begin{verbatim}
    $ ssh -p 77 [username]@[servername].sfs.uni-tuebingen.de
\end{verbatim}

\section{Backups}

The following servers have automatic backups scheduled that should be cleared regularly:

\begin{enumerate}

\item Wagner
\begin{itemize}
\item Automatically purged after available disk space falls below 5GB
\end{itemize}

\item Haendel
\item Beethoven
\item WWW2 (Vivaldi)
\item Scarlatti

\end{enumerate}



\section{Scheduled tasks}

There are some scheduled tasks on Wagner, specifically the backup and cleanup scripts, that can only be accessed by first switching to root.

\begin{verbatim}
    $ sudo su        // switch to root
    $ crontab -e     // display scheduled tasks
\end{verbatim}



\section{Display problems}

https://unix.stackexchange.com/questions/292042/how-to-fix-shell-prompt-ps1-odd-escape-sequences-after-remote-login-from-ite \\

\noindent
https://askubuntu.com/questions/846627/weird-character-appeaaring-when-connecting-another-server-over-ssh-on-ubuntu-16 \\

\noindent
https://www.iterm2.com/documentation-shell-integration.html

\section{SVN accounts}

When a new Subversion account is needed, you should first generate a secure password by using the \verb|pwgen| command:

\begin{verbatim}
    $ pwgen
\end{verbatim}

\noindent
Then, simply navigate to:

\begin{verbatim}
    $ cd /var/local/svn/svn.wagner.sfs.uni-tuebingen.de/conf
\end{verbatim}

\noindent
and modify the \verb|passwd| file, followed by the secure password from \verb|pwgen|:

\begin{verbatim} 
    $ htpasswd -m passwd [username]
\end{verbatim}

\noindent
Finally, add the new username to the list of authorized users in the \verb|authz| file, at the end of "svngroup" line:

\begin{verbatim}
    $ nano authz
\end{verbatim}

\noindent
If you wish to remove a user's access, simply remove their username from the "svngroup" line, and add it to "removed users" for bookkeeping.

\end{document}
